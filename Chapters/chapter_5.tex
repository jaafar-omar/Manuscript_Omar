\renewcommand{\thechapter}{\Roman{chapter}}
\chapter{Conclusion and Recommendation}
\renewcommand{\thechapter}{\arabic{chapter}}
\label{ch:Result and Discussion}
\thispagestyle{empty}

\section{Conclusion}
\label{ch5:sec:conlusion}
In conclusion, the conducted tests provided valuable insights into the performance of the developed system. It is important to note that these tests were carried out under controlled conditions, ensuring the absence of dust clouds in the empty space to facilitate accurate surface scanning of the flour product by the LiDAR. This controlled environment helped to validate the system's ability to measure volume accurately under specific conditions.

Based on the summarized results presented in the previous chapter, the system exhibited considerable performance in accurately measuring the volume of the flour storage bins. With an average Mean Absolute Percentage Error (MAPE) of 0.909655\%, the system demonstrated a relatively low level of error, remaining below the 1\% threshold. The observed errors ranged from 0.5993776\% to 1.27379\%, indicating consistency across different storage level capacity.

\section{Recommendations}
In future studies, it may be consider using a LiDAR systems with multi-echo functionality. Such systems can effectively penetrate dust and other particles, allowing for accurate scanning even in challenging environmental conditions. Additionally, considering the computational limitations of the Raspberry Pi, researchers may opt for more powerful single-board computers to enhance processing speed and efficiency. By addressing these potential areas for improvement, future iterations of the system could offer enhanced performance and reliability in real-world applications.