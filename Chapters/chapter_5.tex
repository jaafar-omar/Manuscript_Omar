\renewcommand{\thechapter}{\Roman{chapter}}
\chapter{Conclusion and Recommendation}
\renewcommand{\thechapter}{\arabic{chapter}}
\label{ch:Result and Discussion}
\thispagestyle{empty}

\section{Conclusion}
\label{ch5:sec:conlusion}
% In conclusion, the conducted tests provided valuable insights into the performance of the developed system. It is important to note that these tests were carried out under controlled conditions, ensuring the absence of dust clouds in the empty space to facilitate accurate surface scanning of the flour product by the LiDAR. This controlled environment helped to validate the system's ability to measure volume accurately under specific conditions.

% Promising results from the testing of the 3D Point Cloud Scanner System showed that it was capable of measuring the volume of flour materials in the storage bin with accuracy with minimal standard deviations and Mean Absolute Percentage Errors (MAPEs) throughout the test. The system's accuracy and precision were further validated by volume measurement tests, which showed minimal variation between measured and actual volumes. The results demonstrate the system's potential for useful applications in monitoring storage capacity by showing the system can accurately estimate flour amounts under a variety of conditions.

The study successfully designed and developed a web-based 3D point cloud scanner system capable of measuring the product volume inside a flour storage bin. The integration of the 3D point cloud scanner system and web application achieved the objectives. Promising results from testing the 3D Point Cloud Scanner System showed that it was capable of measuring the volume of flour materials in the storage bin with accuracy and the calibration result with statistical analysis emphasizing that the measured volume of the system has no significant difference between the known volume of the storage bin throughout the test. The system's accuracy and precision were further validated by comparing the volume measurements of the system to those of the traditional method, revealing a significant difference between the two methods. This suggests that the developed system is more accurate than the traditional method. Overall, the results demonstrate the system's potential for useful applications in monitoring storage capacity, as it can accurately estimate flour amounts under a variety of conditions.


\section{Recommendations}
In future studies, it may be consider using a LiDAR systems with multi-echo functionality. Such systems can effectively penetrate dust and other particles, allowing for continuous scanning even dust is present. Additionally, considering the computational limitations of the Raspberry Pi, researchers may opt for more powerful single-board computers to enhance processing speed and efficiency. By addressing these potential areas for improvement, future iterations of the system could offer enhanced performance and reliability in real-world applications.