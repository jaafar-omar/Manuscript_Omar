\begin{center}
	\noindent \textbf{ABSTRACT}
\end{center}

\addcontentsline{toc}{chapter}{\MakeUppercase{abstract}\vspace{-0.5cm}}
%Monitoring the product inside a large storage bin in a food manufacturing industries, particularly in industries dealing with essential commodities like flour is of utmost significance for various reasons. Ensuring accurate and timely monitoring of flour storage bins prevents detrimental scenarios such as underproduction or overstocking. In the case of underproduction, inadequate monitoring leading to stockouts can disrupt the production process, resulting in a potential profit loss due to client charge. Conversely, overstocking can lead to increased risk of product spoilage or infestation. Specifically, in the context of flour storage, overstocking can attract flour beetles when left unsold, leading to infestation and contamination of the stored flour. Infestation in both finished flour products and raw material grains (wheat) poses a significant challenge for the company at present. Internal and external failure costs related to infestation complaints accounted for around 3.60\% of the total Cost of Quality for flour products recorded by some food manufacturers during the period of January to December 2018. Another factor of such inadequate monitoring is the manual volume measurement which is labor-intensive and prone to error.

Adequate monitoring of flour storage bins in the food manufacturing industry is crucial to prevent profit loss from underproduction and overstocking. Underproduction can lead to stockouts and profit loss, while overstocking increases the risk of spoilage and contamination by flour beetles. Infestation in products and raw materials can cost manufacturers up to 3.60\% of the total Cost of Quality. Additionally, manual volume measurement is labor-intensive and error-prone, highlighting the need for efficient monitoring.

This study presents the design and development of volumetric measurement using Web-based 3D Point Cloud Scanner System attached at the top of storage bin to scan and automatically measure the empty space and product volume inside of a storage. The successful integration and actual testing of the system highlighted the effectiveness and efficacy in a real-world setting. The inclusion of the web application provided functionalities such as remote scanning and volume monitoring to address the manual and laborious measurement method.

\vspace{1cm}

\noindent\textit{\textbf{Keywords}}\textemdash automation, web application, ROS, point cloud scanner, volume measurement

% The main purpose of the web-based application is to eliminate the manual process of volume measurement that requires personnel to physically monitor the storage. The experimental approach involved conceptualizing and prototyping the 3D-PCSS and the web-based application, followed by different testing scenarios and evaluating the performance of the overall system. The testing scenarios involved scanning the empty storage bin and comparing it to the actual volume of the storage bin based on its geometric shape to establish the ground truth of the maximum volume capacity. Furthermore, the storage bin was manually filled with known volumes of flour, and the system measured the volume for direct comparison. The error metric was determined by comparing the system's measurements with the actual volumes at varying storage percentage capacities: 14.62\%, 42.39\%, and 70.76\% of the storage maximum volume capacity. The experimental result of the study shows that 0.5307\% error was achieved when measuring the empty storage. Additionally, testing at varying storage percentage capacity levels yielded percentage errors of 1.1572\%, 0.9445\%, and 1.7763\% respectively, averaging 1.309\% error. These findings underscore the system's reliability and potential for efficient volume measurement, validating its suitability for practical application in storage management contexts.